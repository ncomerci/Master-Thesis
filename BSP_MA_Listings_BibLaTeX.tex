%
% FH Technikum Vienna
% !TEX encoding = UTF-8 Unicode
%
% Creation of Master and Bachelor theses at the FH Technikum Wien using LaTeX and the TWBOOK class
%
% To create your own document you have to add the following:
% 1) Set[...] with \documentclass: Master's or Bachelor's thesis, course of study and language.
% 2) With \newcommand{\FHTWCitationType}. Set citation standard (is usually given by the study program - please ask)
% 3) fill in cover sheet, abstract, etc.
% 4) and write the paper (enter the used literature sources in Literatur.bib)
%
% Tested with TeXstudio with character encoding ISO-8859-1 (=ansinew/latin1) and MikTex under Windows
% Note that the encoding of the file matches the encoding of the package inputenc!
% The encoding of the file twbook.cls MUST be ANSI!
% When using UTF8 not only the encoding of the document must be set to UTF8, but also the encoding of the BibTex file!
%
% Please send bugreports and feedback via email to latex@technikum-wien.at
%
% Versions
% *) V0.7: 9.1.2015, RO: modeline adjusted and moved
% *) V0.6: 10.10.2014, RO: Further adaptation to UK
% *) V0.5: 8.8.2014, WK: Literature sources revised and adapted
% *) V0.4: 4.8.2014, WK: Inital version imported into SVN
%
\documentclass[MSE,Master,english]{twbook}%\documentclass[Bachelor,BMR,ngerman]{twbook}
\usepackage[utf8]{inputenc}
\usepackage[T1]{fontenc}
\usepackage{csquotes}

\tolerance=1
\emergencystretch=\maxdimen
\hyphenpenalty=10000
\hbadness=10000

%
% Hier biblatex & Biber konfigurieren; Vergessen Sie nicht, dass Sie biber verwenden müssen um eine Bibliothek zu erzeugen
%
\usepackage[backend=biber, style=numeric]{biblatex}
\addbibresource{Literatur.bib}

%
% Bei Bedarf bitte hier die Syntax-Highlightings anpassen
%
\usepackage[final]{listings}
\lstset{captionpos=b, numberbychapter=false,caption=\lstname,frame=single, numbers=left, stepnumber=1, numbersep=2pt, xleftmargin=15pt, framexleftmargin=15pt, numberstyle=\tiny, tabsize=3, columns=fixed, basicstyle={\fontfamily{pcr}\selectfont\footnotesize}, keywordstyle=\bfseries, commentstyle={\color[gray]{0.33}\itshape}, stringstyle=\color[gray]{0.25}, breaklines, breakatwhitespace, breakautoindent}
\lstloadlanguages{[ANSI]C, C++, [gnu]make, gnuplot, Matlab}

\usepackage[xindy]{glossaries}
\makenoidxglossaries
\renewcommand{\glossarysection}[2][]{}
%Formatieren des Quellcodeverzeichnisses
\makeatletter
% Setzen der Bezeichnungen für das Quellcodeverzeichnis/Abkürzungsverzeichnis in Abhängigkeit von der eingestellten Sprache
\providecommand\listacroname{}
\@ifclasswith{twbook}{english}
{%
    \renewcommand\lstlistingname{Code}
    \renewcommand\lstlistlistingname{List of Code}
    \renewcommand\listacroname{List of Abbreviations}
}{%
    \renewcommand\lstlistingname{Quellcode}
    \renewcommand\lstlistlistingname{Quellcodeverzeichnis}
    \renewcommand\listacroname{Abkürzungsverzeichnis}
}
% Wenn die Option listof=entryprefix gewählt wurde, Definition des Entyprefixes für das Quellcodeverzeichnis. Definition des Macros listoflolentryname analog zu listoflofentryname und listoflotentryname der KOMA-Klasse
\@ifclasswith{scrbook}{listof=entryprefix}
{%
    \newcommand\listoflolentryname\lstlistingname
}{%
}
\makeatother
\newcommand{\listofcode}{\phantomsection\lstlistoflistings}

% Die nachfolgenden Pakete stellen sonst nicht benötigte Features zur Verfügung
\usepackage{blindtext}
%
% Einträge für Deckblatt, Kurzfassung, etc.
%
\title{Transparency in\\Decentraland DAO}
\author{Comerci Wolcanyik, Nicol{\'a}s}
\studentnumber{2120299002}
%\author{Titel Vorname Name, Titel\and{}Titel Vorname Name, Titel}
%\studentnumber{XXXXXXXXXXXXXXX\and{}XXXXXXXXXXXXXXX}
\supervisor{Dinhof, Gerhard}
%\supervisor[Begutachter]{Titel Vorname Name, Titel}
%\supervisor[Begutachterin]{Titel Vorname Name, Titel}
%\secondsupervisor{Titel Vorname Name, Titel}
%\secondsupervisor[Begutachter]{Titel Vorname Name, Titel}
%\secondsupervisor[Begutachterinnen]{Titel Vorname Name, Titel}
\place{Vienna}
\outline{\toComplete}
\keywords{Keyword1, Keyword2, Keyword3, Keyword4}
\acknowledgements{\toComplete}

% Glossary entries
\newglossaryentry{MANA}
{
  name=MANA,
  description={Is Decentraland's fungible, ERC20 cryptocurrency token limited to a total original supply of 2,805,886,393},
}
\newglossaryentry{LAND}
{
  name=LAND,
  description={Is a scarce, non-fungible digital asset maintained in an Ethereum smart contract that represents the parcels of virtual land within Decentraland},
  plural=LANDs
}
\newglossaryentry{ESTATE}
{
  name=ESTATE,
  description={A cluster of adjacent LANDs},
  plural=ESTATEs
}
\newglossaryentry{NFT}
{
  name=NFT,
  description={non-fungible tokens are unique, distinguishable digital assets. The information contained within a non-fungible token is unique to that token},
  plural=NFTs
}

\begin{document}

\maketitle

%
% .. und hier beginnt die eigentliche Arbeit. Viel Erfolg beim Verfassen!
%
\chapter{Introduction}
From October 31, 2008 (when the Bitcoin whitepaper was published) to the present day, people's interest in Blockchain technology has been growing progressively and increased exponentially from 2020 onwards. 

There are a large number of projects that rely on this technology to date, one of them is Decentraland: a 3D digital world that will be discussed in the next section and that uses Ethereum to execute its transactions.

One of the great advantages of Blockchain is transparency: anyone can verify all transactions made, from the first to the last. But how easy is it for an average user to access this information? What kind of analysis and conclusions can be drawn once obtained? This paper will seek to answer these questions specifically using Decentraland's DAO as the object of study. \\
\toComplete
\clearpage

\chapter{Basics}
\section{Blockchain}
Today there are different types of blockchain, but based on the original concept and taking as a reference the article \emph{"La blockchain: fundamentos, aplicaciones y relaci{\'o}n con otras tecnolog{\'\i}as disruptivas"}\cite{blockchain}, it was created to store the transaction history of Bitcoin, but with the course of time it has seen great potential to be applied in other areas due to the properties it offers. The blockchain provides an immutable distributed database based on a growing sequence of blocks. These blocks, being public, form an open system that enhances trust based on the transparency and robustness of the blockchain construction technique. The system, although open, is also semi-anonymous: users identify themselves with public keys (pseudonyms), not with their real identities.

This database can be shared by a large number of users on a \emph{peer-to-peer} basis and allows information to be stored in an immutable and orderly manner. The information added to the blockchain is public, can be accessed at any time by any user of the network and can only be added to the blockchain if there is an agreement between the majority of the parties. After a certain period of time, it can be assumed that the information added to a block can no longer be modified (immutability).

By design, this system intrinsically provides tolerance to node failures, robustness against manipulation and transparency, since it is public.

\section{Cryptocurrency}
\ac{Crypto} is essentially a digital currency that use blockchain technology and cryptography to facilitate secure and anonymous transactions. The crypto market is worth over USD 500 billion.\cite{crypto}

What made it different from normal bank transfers or other financial services like Paypal is that there was no middle man for the first time. A middle-man is a central authority like a bank or government that intervenes in a transaction between the sender and recipient. They have the power to surveill, censor or revert transactions and they can share the sensitive data they collect with third parties. They also often dictate which financial services people have access to.\cite{ethereum}

\section{Ethereum}
Ethereum is a technology for building apps and organizations, holding assets, transacting and communicating without being controlled by a central authority. There is no need to hand over personal details to use it - the user remains in control of their own data and what is being shared. Ethereum has its own cryptocurrency, Ether, which is used to pay for certain activities on the Ethereum network.

Also it is \textbf{programmable} using \textbf{smart contracts} (see \ref{sm}), so that means that people can build apps that use the blockchain to store data or control what apps can do. This results in a general purpose blockchain that can be programmed to do anything.\cite{ethereum}

\section{Smart contracts\label{sm}}
\emph{"A smart contract is a computerized transaction protocol that executes the terms of a contract. The general objectives of smart contract design are to satisfy common contractual conditions (such as payment terms, liens, confidentiality, and even enforcement), minimize exceptions both malicious and accidental, and minimize the need for trusted intermediaries. Related economic goals include lowering fraud loss, arbitration and enforcement costs, and other transaction costs."} Nick Szabo, 1994.\cite{smartContracts} \\

Smart contracts live on the Ethereum blockchain. They only execute when triggered by a transaction from a user (or another contract). These programs are called \ac{Dapp}.

Once a smart contract is published to Ethereum, it will be online and operational for as long as Ethereum exists. Not even the author can take it down. Since smart contracts are automated, they do not discriminate against any user and are always ready to use.\cite{ethereum}

\section{\NoCaseChange{\acl{DAO}} (DAO)}
A DAO is a collectively-owned, blockchain-governed organization working towards a shared mission.

DAOs allow people to work with like-minded individuals around the globe without trusting a benevolent leader to manage the funds or operations. There is no CEO who can spend funds on a whim or CFO who can manipulate the books. Instead, blockchain-based rules baked into the code define how the organization works and how funds are spent.

They have built-in treasuries that no one has the authority to access without the approval of the group. Decisions are governed by proposals and voting to ensure everyone in the organization has a voice, and everything happens transparently on-chain.\cite{DAO} \\

Table \ref{table:DAOComparison} on page \pageref{table:DAOComparison} compares a DAO with a traditional organization:
\begin{center}
  \begin{table}[htb]
    \begin{tabular}{ | m{20em} | m{20em} | }
      \hline
      \textbf{DAO} & \textbf{Traditional organization} \\ 
      \hline
      Usually flat, and fully democratized. & Usually hierarchical. \\
      \hline  
      Voting required by members for any changes to be implemented. & Depending on structure, changes can be demanded from a sole party, or voting may be offered. \\
      \hline
      Votes tallied, and outcome implemented automatically without trusted intermediary. & If voting allowed, votes are tallied internally, and outcome of voting must be handled manually. \\
      \hline
      Services offered are handled automatically in a decentralized manner (for example distribution of philanthropic funds). & Requires human handling, or centrally controlled automation, prone to manipulation. \\
      \hline
      All activity is transparent and fully public. & Activity is typically private, and limited to the public. \\
      \hline
    \end{tabular}
    \caption{Comparison between a DAO and a traditional organization}
    \label{table:DAOComparison}
  \end{table}
\end{center}

\section{Decentraland\label{dcl}}
Decentraland is a decentralized virtual reality platform powered by the Ethereum blockchain. It is an open source project, maintained by the Decentraland Foundation and driven by its community. 

Within the Decentraland platform, users can create, experience, and monetize their content and applications. The finite, traversable, 3D virtual space within Decentraland is called \textbf{\gls{LAND}}, a non-fungible digital asset or more commonly known as a \ac{NFT}, maintained in an Ethereum smart contract. Land is divided into parcels that are identified by cartesian coordinates (x,y). These parcels are permanently owned by members of the community and are purchased using \textbf{\gls{MANA}}, Decentraland's cryptocurrency token. This gives users full control over the environments and applications that they create, which can range from anything like static 3D scenes to more interactive applications or games.\cite{DCL}

The Decentraland DAO is the decision-making tool for digital assets holders in Decentraland's virtual world. Through votes in the DAO, the community can issue grants and make changes to the lists of banned names, POIs, and catalyst nodes. The DAO also controls the \gls{LAND} and \gls{ESTATE} smart contracts.\cite{DCLDAO}

\chapter{Problem}
As already mentioned in section \ref{dcl}, Decentraland is open source and community driven. Based on this, where many actors are involved in decision making, it is essential to have as much information as possible about the status of the project in order to make informed decisions.

The problem was that there was no tool or solution that could publicly provide information to everyone in semi-real time about the state of the DAO treasury and the different assets of the community. Of course the information was and still is available since the blockchain is public but not everyone has the time or knowledge to collect such data and analyze it.

\chapter{Discussion}
\toComplete

\chapter{Conclussion}
\toComplete

\chapter{Summary}
\toComplete

\chapter{Lists}
\toComplete

\chapter{Glossary}
% \glossarysection
\printnoidxglossaries

\chapter{Appendix (optional)}
\toComplete

%
% Hier beginnen die Verzeichnisse.
%
\clearpage
\printbibliography
\clearpage
% Das Abbildungsverzeichnis
\listoffigures
\clearpage

% Das Tabellenverzeichnis
\listoftables
\clearpage

% Das Quellcodeverzeichnis
\listofcode
\clearpage

\phantomsection
\addcontentsline{toc}{chapter}{\listacroname}
\chapter*{\listacroname}
\begin{acronym}
    \acro{Crypto}{Cryptocurrency}
    \acro{Dapp}{Decentralized App}
    \acro{DAO}{Decentralized Autonomous Organization}
    \acro{NFT}{Non-Fungible Token}
\end{acronym}

%
% Hier beginnt der Anhang.
%
\clearpage
\appendix
\end{document}
